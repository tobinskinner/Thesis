\setcounter{footnote}{0}
\setcounter{section}{0}
\setcounter{ExNo}{0}
\section*{}
\addcontentsline{toc}{section}{Acknowledgements}
\singlespacing
\mbox{\sc{Acknowledgements}}\\\\
First and foremost, I would like to thank my thesis committee: Lisa Travis, Glyne Piggott, and Jon Nissenbaum. Collectively, I could not have asked for a better or more balanced group of advisers, in terms of both academic expertise and advising style. Individually, they have pushed, guided, and encouraged me unfailingly throughout my time at McGill. If it were not for each of them, you wouldn't be reading this right now.

I thank Jon for showing me early on how utterly exciting theoretical syntax can be and, above all, for constantly challenging me to defend my ideas and consider the big picture. Jon taught me that a little give and take can go a long way. I always left our meetings with a renewed sense of accomplishment, as he often led me to realize that my proposals had many more (usually good) repercussions than I had previously thought.

A special thanks to Glyne, who showed me that, although syntax and phonology are both interesting pursuits on their own, nothing compares to studying them side-by-side. This entire thesis is the progeny of a term paper I wrote on reduplication for one of Glyne's advanced phonology courses. He saw something in my ideas that I hadn't. Though I was hesitant at first to make any strong claims, he convinced me that I had something important to say, and he pushed me --and continues to push me-- to refine those ideas and to share them with the larger linguistics community. Slowly but surely, I have come to think that Glyne was right. In short, Glyne has made me believe in myself as a linguist.

A very special thanks to Lisa, my supervisor.  She has been the backbone of this entire process, and I could not have done it without her. Whenever I was feeling lost or doubting my research, I would schedule an emergency meeting with Lisa because I knew that she would set me back on the right track by putting things in perspective and reminding me why my work was worthwhile. In my occasional struggles between syntax and phonology, it was always Lisa who helped me to find a middle ground. She kept me sane throughout the writing of this thesis, and always shared in the excitement of what I was doing. She is a model that I aspire to, not only as an academic but also as a person, and I thank my lucky stars that I've had the opportunity to work with her.

I'd also like to thank some of the other professors I've had in my seemingly interminable career as a student. Thanks to my undergraduate adviser at the University of Virginia, Ricardo Padr\'{o}n. Though I've come a long way from queer theory in medieval Spanish literature to syntactic theory in modern linguistics, I'm often reminded of the example he set for me as a professional academic. Thanks to Ellen Contini-Morava, whose \textit{20$^{th}$ Century Linguistic Theory} class, which I took during my final year at UVA, made me come to the sudden, life-altering realization that linguistics was the right field for me. Special thanks to Paul Hagstrom, my M.A. thesis adviser at Boston University. Under Paul's guidance, I discovered that it was actually possible for me to be a successful linguist, and that achieving this success does not mean that one has to give up little pleasures like {\it The Simpsons}. For many reasons, I would not be where I am today were it not for him, and for that I am eternally grateful. Thanks to Shanley Allen, who, as an outstanding professor and a McGill graduate, showed me that this school can produce excellent linguists. I hope I can live up to her example. Thanks to Kyle Johnson, my first --and perhaps most memorable-- professor at McGill. I wish he had been able to stick around, despite his tendency to hurl pieces of chalk at his students. Thanks to Heather Goad, not only for teaching me phonology, but also for exemplifying what a teacher should be. Any success that I have had as a teacher is due only to the fact that I attempt to mimic her. Thanks to Bernhard Schwarz for being an excellent adviser during my dalliances with the syntax-semantics interface and for helping me to keep the butterflies at bay when teaching in front of 150 students for the first time. Thanks to Yosef Grodzinsky, who, like me, was under the false impression that I would eventually become an acquisitionist when I arrived at McGill, but who supported my decision to pursue other avenues of research. Thanks to Lydia White for always being available to help me with judgments on the strange, exotic language known as British English. And thanks to Brendan Gillon, who, though we did not have a chance to interact too often, reminded me to never take anything at face value. Thank you also to the attendees of what is now known as the McGill Syntactic Interfaces Research Group (McSIRG), especially Maire Noonan, Junko Shimoyama, and Michael Wagner; and to the \textit{Fonds qu\'{e}b\'{e}cois de recherche sur la soci\'{e}t\'{e} et la culture} for McSIRG's government grant FQRSC 2010-SE-130906. And thank you to Andr\'{e}s Pablo Salanova and David Embick for helping to make the oral defense of this thesis an incredibly fun and intellectually stimulating experience.

Thank you to the support staff of the Linguistics department, especially Andria De Luca and Connie DiGiuseppe, who always provided help when I most needed it.

Thank you to some of the other students in my cohort, who have been with me from my first day at McGill: Eva Dobler for becoming one of my most favorite research partners and essentially my twin sister (give or take one year exactly); Deena Fogle for making me feel less crazy when I talk about Vinny and Mario as if they were people, not cats, and for rescuing me at parties; and Gustavo Beritognolo, whose friendship helped to make McGill feel like home from the moment I got here.

A big thanks to Raph Mercado, who served as my big brother in linguistics for the longest time (though we're the same age) and is still a dear friend. My research would never have gotten off the ground without his help and example. {\it Maraming salamat}! Thanks to Joey Sabbagh for teaching me about the linguistics world beyond McGill during his time here and for being an incredible friend and first-class drinking buddy.

Thanks to \"{O}ner \"{O}z\c{c}elik for not laughing too hard when I tried to pronounce Turkish, and to Katarina Smedfors for not questioning why a linguist would be asking such strange questions about Swedish. Thanks also to David-\'{E}tienne Bouchard not only for his help with French, but also for the many discussions on syntax-phonology-semantics that all seemed to just make sense.

A heartfelt thanks to Heather Newell, whose footsteps I follow in. The current research would not even be feasible were it not for the precedent she set with her own work. I am proud to call her my colleague, but even prouder to call her my friend. Luckily for me, she finished well before I did. Thanks to Mina Sugimura, in whom I've found a kindred spirit and who is not allowed to go back to Japan ever. Thanks to Andrea Santi for always being there when I needed her and never kicking me out of her office even though she had work to do. And thanks to Nino Grillo, whose smile can light up the night sky.

Thank you to my other fellow grad students, both past and present, who have helped me to get where I am in one way or another: Naoko Tomioka, \'{E}mile Khordoc, Thanasis Tsiamas, Moti Lieberman, Walter Pedersen, Bethany Lochbihler, Jozina Vander Klok, Alan Bale, and Larissa Nossalik (who beat me by a photo finish!). Thank you also to my students, who keep me honest, and who have consistently reminded me that it is my love of teaching that sent me down this road to begin with.

I'd also like to thank the clientele of the (now defunct) karaoke bar Agora. You endured my less-than-sober renditions of classic \textit{qu\'{e}b\'{e}cois} pop songs night after night, and for that you should be acknowledged. Thank you also to Luke Windisch, Mike McTague, and Melissa Willey for daring to befriend a linguist, even though all we do is analyze the way you speak. Additionally, thank you to the makers of the following products: \LaTeX, for making the formatting of this thesis a snap; the World of Warcraft, for giving me something to do when I wasn't dissertating; and Moosehead beer, on which you can blame any typos.

Finally, I'd like to thank my family, who have supported me my entire life and are undoubtedly thrilled that I can finally say ``I'm done". To my mother, Ann Whitley, and my father, Rhae Skinner, thank you for making me the person that I am today. To my big sister, Wendy, thank you for always being a shoulder I can lean on. To my step-sister, Brittany Phippins, for being the little sister I didn't know I always wanted. To my step-mother, Carol Skinner, who has been my friend through the most trying of times. To Nickole and Dave Hannah and Jordan Horner for being the best in-laws a guy could hope for. To my cousin, Laurie Skinner, who made my adolescence in southeast Virginia bearable; to my aunt Georgianna Skinner, for showing me at a young age that being ``book smart'' is more than okay; to Carolyn Skinner, for being the coolest aunt in the world; to Michael Whitley, for putting up with me when I was a rebellious teenager. And to my grandmother, Gladys Douglas Hackworth, who has been one of my greatest supporters and whose extraordinary life I hope to emulate.

Thank you most to my husband, Josh Horner. You have been there every step of the way with me, and your love and support have sustained me. This is for us.

\onehalfspacing