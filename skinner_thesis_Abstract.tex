\setcounter{footnote}{0}
\setcounter{section}{0}
\setcounter{ExNo}{0}
\section*{}
\addcontentsline{toc}{section}{Abstract}
\singlespacing
\begin{center}
\textbf{Investigations of downward movement}
\end{center}
\mbox{\sc{Abstract}}\\\\
Under a non-lexicalist view of word formation, such as Distributed Morphology \citep{halle_marantz1993}, morphemes combine to form complex words during or after---but not before---narrow syntactic derivation. Such a model inevitably requires the availability of downward transformations, e.g.\ affix-hopping. This thesis provides a detailed investigation into such downward movements. Whereas previous analyses have relegated downward movements to a position outside of core derivational processes (e.g.\ \citenop{chomsky1981} and, to a lesser extent, \citenop{embick_noyer2001}), I argue that certain downward movements, namely head-to-head Lowering, form part of the central architecture of syntactic derivation and are motivated by fundamental properties of that architecture, such as phase impenetrability \citep{chomsky2001}.

Though this thesis addresses certain properties of other types of apparent downward movement (e.g.\ morpho-phonological merger; i.e.\ Local Dislocation), it focuses primarily on the defining characteristics of head-to-head Lowering. Central to this investigation is the observation that Lowering is a highly syntactic operation. In Chapter 2, I argue that a Lowering head may freely target any intermediate syntactic position of the complex head of its complement, thus deriving several cases of morphological optionality; e.g.\ reduplicative variability in Tagalog and Ndebele and the variable positions of agreement markers in Turkish.

Chapter 3 addresses tense-hopping, a canonical case of downward movement. I argue that certain asymmetries between English and Swedish provide evidence that these two languages derive their respective tense-hopping patterns via different means. Namely, Swedish tense-hopping is a case of Lowering, whereas English tense-hopping results from Local Dislocation (following \citenop{ochi1999}). Additionally, I propose a detailed theory of the Lowering vs.\ Raising distinction. Based in the observation that Lowering only ever takes place across a phase boundary, I posit a {\it Phase Head Impenetrability Condition} (PHIC), under which features embedded in a complex phase head become inaccessible as a result of Spell-out. Lowering occurs as a last resort feature-checking operation when the next highest head targets one of these embedded features; Raising occurs otherwise. I address several repercussions of this analysis, and in Chapter 4 I show that the PHIC allows for a straightforward account of the aux-raising vs.\ tense-hopping asymmetry in English. More precisely, I claim that auxiliary verbs are merged in the same phase as finite tense, and so the PHIC does not apply between these two elements, unlike with main verbs.

The analyses presented in this thesis all share a common goal: to show that downward movements can and should be incorporated into core linguistic theory.
\clearpage
\noindent
\mbox{\sc{R\'{e}sum\'{e}}}\\
Selon une conception non-lexicaliste de la formation du mot, telle que la Morphologie Distribu\'{e}e (Halle et Marantz 1993),  les morph\`{e}mes se combinent pour former des mots complexes pendant ou apr\`{e}s---mais pas avant---la d\'{e}rivation syntaxique \'{e}troite. Un tel mod\`{e}le requiert in\'{e}vitablement la disponibilit\'{e} de transformations descendantes, par exemple la transformation affixale.  La pr\'{e}sente th\`{e}se procure une investigation d\'{e}taill\'{e}e de tels mouvements descendants. Tandis que les analyses pr\'{e}c\'{e}dentes ont rel\'{e}gu\'{e} les mouvements descendants hors du c\oe ur des processus d\'{e}rivationnels (p.\ ex., Chomsky 1981 et, dans une moindre mesure, Embick et Noyer 2001), je soutiens que certains mouvements descendants, soit l'abaissement t\^{e}te-\`{a}-t\^{e}te, forment une partie de l'architecture centrale de la d\'{e}rivation syntaxique et sont motiv\'{e}s par des propri\'{e}t\'{e}s fondamentales de cette architecture, tel que l'imp\'{e}n\'{e}trabilit\'{e} phasique (Chomsky 2001).

Bien que la pr\'{e}sente th\`{e}se examine certaines propri\'{e}t\'{e}s d'autres types de mouvements descendants apparents (p.\ ex., fusionnement morpho-phonologique, c'est-\`{a}-dire, la Dislocation Locale), elle se concentre principalement sur les caract\'{e}ristiques d\'{e}finitionnelles de l'abaissement t\^{e}te-\`{a}-t\^{e}te. L'observation que l'abaissement est une op\'{e}ration proprement syntaxique est centrale \`{a} cette investigation. Au chapitre 2, je soutiens qu'une t\^{e}te abaissante peut librement cibler toute position syntaxique de la t\^{e}te complexe de son compl\'{e}ment, d\'{e}rivant ainsi plusieurs cas d'optionalit\'{e} morphologique, par example, la variabilit\'{e} reduplicative en tagalog et ndebele et les positions variables des marqueurs d'accord en turc.

Le chapitre 3 examine la transformation du temps. Je soutiens que certaines asym\'{e}tries entre l'anglais et le su\'{e}dois fournissent la preuve que ces deux langues d\'{e}rivent leurs patrons de transformation du temps respectifs par des moyens diff\'{e}rents. Plus pr\'{e}cis\'{e}ment, la transformation de temps en su\'{e}dois est un cas d'abaissement, tandis que la transformation du temps en anglais r\'{e}sulte d'une Dislocation Locale (selon Ochi 1999).  De plus, je propose une th\'{e}orie d\'{e}taill\'{e}e de la distinction entre l'abaissement et la mont\'{e}e. Sur la base de l'observation que seul l'abaissement ne prend place qu'\`{a} travers une fronti\`{e}re phasique, je stipule une condition d'imp\'{e}n\'{e}trabilit\'{e} de la t\^{e}te phasique (PHIC), sous laquelle les traits enchass\'{e}s dans une t\^{e}te de phase complexe deviennent inaccessibles \`{a} la suite de l'\'{e}pel. L'abaissement prend place comme op\'{e}ration de v\'{e}rification de trait en dernier recours lorsque l'avant-derni\`{e}re t\^{e}te cible un de ces traits enchass\'{e}s; autrement, la mont\'{e}e se produit. J'explore plusieurs r\'{e}percussions de cette analyse, et au chapitre 4 je montre que la PHIC permet une explication directe de la mont\'{e}e des auxiliaires par opposition \`{a} la transformation du temps en anglais. Plus pr\'{e}cis\'{e}ment, je soutiens que les verbes auxiliaires et le temps fini sont fusionn\'{e}s dans la m\^{e}me phase, et ainsi la PHIC ne s'applique pas entre eux, contrairement aux verbes principaux. Les analyses pr\'{e}sent\'{e}es dans cette th\`{e}se partagent toutes un but commun : montrer que les mouvements descendants peuvent et doivent \^{e}tre incorpor\'{e}s dans la th\'{e}orie linguistique fondamentale.

\onehalfspacing